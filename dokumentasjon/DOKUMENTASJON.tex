%\documentclass[a4paper,norsk,11 pt]{article}             
\documentclass[norsk,11 pt]{report}             

\usepackage[a3paper]{geometry}
%\usepackage[a4paper, landscape]{geometry}

\usepackage[utf8]{inputenc}
\usepackage[norsk]{babel}
\usepackage[T1]{fontenc}

\usepackage{verbatim}

\title{Dokumentasjon for Auron}
\author{Per R. L.}
\date{\today}     
\begin{document}   

%\maketitle

% for å inkludere en seksjon som er utdelt i smådokument: \input{ dokumentNavn }

\chapter{Introduction}

Tester litt \emph{generellt} om \LaTeX \footnote{jejeje, fuck'z off'z!} \\
\- her og?
\begin{description}
\item[første] Første element..  slutt
\item[andre] Andre element.
\end{description}





\section{TID}
For å ordne tid, slik at det er avhengig av global tid, eller klokketikk i prosessor, eller andre urelevante faktore, har eg implementert en sceduler for
Auron. Denne går ut på at ei arbeidsliste holder orden på kva jobber som står for tur. Kvar gang en jobb blir utført, og denne leder til nye jobber,
blir desse nye jobbane lagt til på slutten av lista. I tillegg har eg eit objekt som har samme type "jobbane" i lista: synapse. 
	\\Jobbane er av type synapse, og synSkilleElement er av typen synSkilleElement : public synapse ; 

Videre er funskjonen void regnUt() overloada, slik at ulike handlinger utføres når *arbeidslista->regnUt() kalles.
Desse funskjonane er:
\begin{itemize} 	%eller [itemize] i berland-dokument.. Sjekk og [descripiton]
\item	Øker tid med en. (Tid er en integer med discrete hopp).
\item	Legger peiker til seg selv (objektet) til bakerst på arbeidslista.
\item Sjekker om noken av neuroSensor-ane skal fyre (basert på frekvensen de skal fyre med, som igjen skal være basert på tilstand til måleverdi)
\end{itemize}

Eg hadde i utgangspunktet tenkt å implementere ei arbeidshistorie-liste også, som skulle brukes til å beregne frekvens og lignande, men gjekk vekk fra dette
siden eg ikkje fikk til en tilfredsstillande "momentan-frekvens". Dette er forsåvidt (nesten) en selvmotsigelse. Vanskelig.
Istadenfor går løsninga mi ut på å vente i 1/frekvens tid mellom kvart signal. Dette er nermaste eg kommer momentanfrekvens.


\section{NEURON:}


\section{SYNAPSE:}
Synapse ligger mellom to neuron, som det gjør i virkeligheten. Kvart synapse-objekt har to neuron-peikera, en til presyn. og en til postsyn.
Er ikkje sikker på korleis dette opprettholdes, eller blir initiert/destruert. Finn ut og skriv dette avsnittet.
	\subsection{SYN DEPRESSION: (short time)}
	Har implementert ulAntallSynapticVesiclesAtt, og at denne er viktig for overføring av signal til postsyn. neuron. Det som blir sendt over, blir 
	trekt fra synapsens beholdning. Det er prosentvis sending av syn.vesicles, så når synapsen er halvtom, vil halvparten av signalet sendes.
	Reproduksjon av syn. vesicles er dynamisk ved variabelen dOppladingsFartForSynVesicles (for tida ikkje dynamisk (i endring), men dette er plan.)
	
	Dette er med andre ord (short-time) presynaptisk plasticitet.


	\subsection{LTP og homoLTD:}
	Akkurat no driver eg å sysler med LTP og homoLTD. Dette har eg tenkt å implementere fra synapsens regnUt()-funksjon. Når den er ferdig med å regne ut
	en verdi, skal dette sendes inn i postsynaptisk neuron. Funksjonen for dette returnerer en integer for kor depolarisert den er. Dette har eg planer 
	om å ta med i avgjersla om LTP og homoLTD skal initieres. 
		\subsubsection{homoLTD:}
		Dette er foreløpig en passiv funkson der vekta "lekker litt", vurderer å gi den økande derivert (lekke raskare og raskare ved påfølgande LTD)
		Vektendring går i første omgang på dGlutamatReceptoreIPostsynMem, og enventuelt etter kvar på min teori om 
		antall synapticVesicles, størrelse på membran, med variabelt setpkt. Men dette kommer nok seinare.
		\subsubsection{STDP:}
		spike time-dependent plasticity. Lær om dette først, så implementer det.

	\subsection{VEKTENDRING:}
		i første omgang går dette på homo-måten ( på samme synapse ). 
		\begin{itemize}
			\item Dersom ei synapse fyrer, og postsyn er kraftig depolarisert, vil LTP oppstå.
			\item Dersom ei synapse fyrer, og postsyn er passiv (ikkje kraftig depolarisert) vil LTD.
			\item LTD vurderer eg å ha med alltid, kvar gang ei synapse er aktiv. Spørsmålet er om eg skal ha konstant LTD-rate, eller om denne skal
			økes ved gjenntatte LTD-kall.

			\item etterkvart STDP.
		\end{itemize}



Har også såvidt begynt med historie av arbeid. Ideen var at jobbane skulle flyttast over hit, når de var gjort. Dette har eg ikkje gjort så mykje med, og er 
ikkje sikker på om det er en god ide, eller ikkje. Greit nok med "momentan frekvens", men kanskje eg har bruk for frekvens-over-lang-tid. Skal prøve å unngå.


%Skriver utgåande om tid, og kva eg har gjort for å få til denne, og optimaliseringa:


\chapter{Tid}

For å ordne tid, slik at det er uavhengig av global tid, eller klokketikk i prosessor, eller andre urelevante faktore, har eg implementert en sceduler for
Auron. Denne går ut på at ei arbeidsliste holder orden på kva jobber som står for tur. Kvar gang en jobb blir utført, og denne leder til nye jobber;
vil denne jobben slettes fra sin plass først i lista, og 
de nye jobbane bli lagt til på slutten av lista. I tillegg har eg eit objekt som har samme type som "jobbane" i lista, synSkilleElement, som itererer tid, og holder styr på periodiske fenomen.
	\\Jobbane er av type synapse, og synSkilleElement er av typen synSkilleElement:public synapse ; (arv) 

Videre er funskjonen void regnUt() overloada, slik at ulike handlinger utføres når *arbeidslista->regnUt() kalles.
Desse funskjonane er:
\begin{itemize} 	%eller [itemize] i berland-dokument.. Sjekk og [descripiton]
\item	Øker tid med en. (Tid er en integer => discrete tids--hopp).
\item	Legger peiker til seg selv (objektet) til bakerst på arbeidslista, og ta bort fra første plass.
\item Sjekker om noken av neuroSensor-ane skal fyre (basert på frekvensen de skal fyre med, som igjen skal være basert på tilstand til måleverdi)
\end{itemize}

Eg hadde i utgangspunktet tenkt å implementere ei arbeidshistorie-liste også, som skulle brukes til å beregne frekvens og lignande, men gjekk vekk fra dette
siden eg ikkje fikk til en tilfredsstillande "momentan-frekvens". Dette er forsåvidt (nesten) en selvmotsigelse. Vanskelig.
Istadenfor går løsninga mi ut på å vente i 1/frekvens tid mellom kvart signal. Dette er nermaste eg kommer momentanfrekvens.


Når det gjelder tid, så har eg også tatt med dette i universet mitt (simuleringsuniverset). Neurona/synapsene er tidsvariante (bruke tid for å lades opp). Denne tida er ikkje avhengig av processor-tics, eller UTC, men av andre ting i simuVerset eg har laga. Løysinga mi var å lage en scheduler, som også lager discrete tidsiterering.

Eg tenkte akkurat no en tanke om å gjøre det meir kontinuerlig. Dette kan eg gjøre ved at kvar jobb blir lagt til i en map<double dTid, synapse* sJobb>, der dTid blir regna ut når jobben blir regna ut. Dette kan gi ei aukande liste for dTid, og kan la dTid være med når eg kalkulerer tidsvariante effekter. Ide! Men foreløpig skal eg tenke på andre ting / utvikle andre aspekter.


\section{Scheduleren}
I utgangspunktet var planen å lage eit system for sekvensen / kausaliteten i Auron-nettet mitt. Møtte raskt på problematikk med å la alle neurona som skulle fyre før neuron A, fyrte, før A.

Foreløpig løysing er basert på ideen om to køer, der kjøringa av de to køene altererer, når ei kø er ferdigkjørt. Køene inneholder jobber, i form av synapse*. Alle arvtakera etter class synapse har (muligens overlagra) funksjonen aktiviserOgRegnUt(). 
\begin{itemize}
	\item I vanlige synapser, vil dette føre til:
		\\ - oppdatering av postsyn. neuron ( oppdatering av [depolarisasjon, DA, anna] på grunnlag av tid siden sist oppdatering), 
		\\ - synaptisk overføring -> EPSP / IPSP
		\\ - LTP/LTD (synaptic plasticity)
	\item I synSkilleElement er aktiviserOgRegnUt() overlagra, og kjører isteden en funk som: 
		\\ - itererer tid. 
		\\ - oppdaterer alle synapser i pNesteSynapseSomIkkjeErFerdigOppdatert.
		\\ - holder orden på alle neuron i pNesteFyringForNeuron (``estimert fyringstid''). ---Dersom ett er ført opp til å fyre neste iterasjon, kjøres neuron::oppdater(), og det legges evt. til i pNesteSynapseUtregningsKoe (og vil fyre på slutten av neste iterasjon). Alternativt blir estimat oppdatert.
\end{itemize}
Eg bruker ikkje to køer, men ei, med eit spesialelement som er nedarva fra synapse og har en del av funksjonane overlagra: synSkilleElement. Det som skjer, er at det ligger eit synSkilleElement på slutten av lista. 
Før man kjem dit, har kanskje nye neuron blitt ført til på lista, og vil fyre til neste tidsiterasjon (-etter synSkilleElement; dvs. i neste tidsiterasjon). Når turen kommer til å kjøre synSkilleElement->aktiviserOgRegnUt(), så vil bl.a. tida øke med eit hakk.

Ein annan ting den holder styr på er kva neuron som skal oppdateres, og kjøre oppdater() på desse. Dersom de fyrer, skal de føres opp på arbeidslista på vanlig måte. Dersom de ikkje fyrer, vil oppdater lage eit nytt estimat av når de fyrer, og føre de opp på estimat-lista (neuron.oppdater() holder orden på denne fyrings-estimat-lista).

Grunnen til å ha ei slik fyrings-estimat-liste, er å optimalisere kjøringa uten å ta bort tids-faktore i neuronet (leaky integrator).
std::map<neuron*, int> sNesteFyringForNeuron heiter estimat-lista.



%Skriver utgåande om neuron, og tidsvariansen i depol. (leaky integrator), og DA (også leaky) - for LTP/LTD balansen.

\chapter{class neuron}

class neuron -- Hovedklasse for Auron. Innholder alt, med unntak av synapse-funksjonaliteten.

\section{Depol-Verdi}
\label{depolVerdi}
\begin{description}\item[neuron::nVerdiForDepolarisering] Inneholder depolarisasjons-verdien til neuronet.\end{description}
Denne verdien kan bli påvirka på fleire måter:
\begin{description}
	\item[(+/--)] Neuronet kan få eksitatorisk eller inhibitorisk input gjennom synapse.
	\item[ (+) ] Neuronet kan egen--eksitere (mekanismen bak ``egenfyring'').
	\item[ (--) ] Neuronet lekker i verdi (``leaky integrator'' -- omtrent som ei bøtte med hull i (mister \%-er kvar runde) ).
\end{description}
Denne kombinasjonen gjør kvart enkelt neuron til eit ganske komplisert system, for ikkje å snakke om fleire i ``team''. Har og forenkla vekk veldig mykje, og har bare berørt toppen av laskatongen av den biologiske varianten. Skal etterkvart auke, med DA(dopamin) som argument til variabel LTP/LTD --likevekt. (Kommer etterkvart).

\subsection{Synaptisk input gir en form for depol-endring}
Når neuronet får EPSP og IPSP, endrer dette pr.def. depolarisasjonen i postsynaptisk neuron. Dette har eg tatt med i Auron.
\begin{description}\item[virtual void sendInnPostsynaptiskEksitatoriskEllerInhibitoriskSignal(int nArg)] 
	Sender inn nArg som E/I input, og neuronets depolarisasjon vil endre seg derretter. Postsynapsens depol. verdi returneres. Ved fyring returneres (-1).
	Timestamp lagres for overføring, og brukes seinere ved ``leaky integrator'' beregning (kor mykje man mister).
	
	Eg har også planlagt å utvide returverdi--opplegget, slik at det går an å signallisere ønske om LTD fra postsyn., men dette er lite gjennomtenkt.
\end{description}

\subsection{Egen--Eksitasjon (gir resultat ``egenfyring'')}
Så vidt eg har forstått det, er egeneksitasjon eit vitig element i neuron. Fører til tonisk egenfyring (umiddelbart viktig bl.a. i DA--system).
Eg har implementert egenfyring i Auron, vha. to--tre mekanismer:
\begin{description}
	\item[void oppdaterNeuron()] --oppdaterer bl.a. neuronets verdi, bl.a. som resultat av egeneksitasjon. Ser og på neuron--lekkasje.
	\item[map<neuron,unsigned> sNesteFyringForNeuron] --holder oversikt over estimert tid for fyring av neuron som resultat av egeneksitasjon. Dette estimatet er alltid for kort, da det ikkje tar hensyn til lekkasjen i neuronet (\%-vis tap av verdi, kvar runde). Dette er bra, siden vi da sjekker verdien for tidlig heller en for seint.\\
Denne lista brukes for å optimalisere systemet, slik at det ikkje blir naudsynt å kjøre oppdaterNeuron() heile tida (spare tid/ressurser).
	\item[tidsSkilleElement->aktiviserOgRegnUt()] ---aktiviserOgRegnUt er overlagra i tidsSkilleElement, slik at isteden for å regne ut postsynaptisk verdi, vil den iterere tid, holde styr på periodiske fenomener og andre ting som varierer med tida; deriblandt depol.--lekkasje og egen--eksitasjon. Ved egen--eksitasjon over terskelen, vil neuronet fyre i tidsSkilleElement, men først etter at tidsSkilleElement har kopiert over peikeren til seg sjølv, over til slutten av arbeidskøa (pNesteSynapseUtregningsKoe). Dette vil føre til at resultatet av fyringa (at synapsene som blir lagt til i arbeidskøa, vil først bli lagt til \emph{etter} neste tidsiterasjon).
\end{description}

\subsection{Lekkasje av verdi, depol--lekkasje}
Egeneksitasjon auker depol.--verdien, lekkasje minker verdien. Korleis henger dette sammen?\\
Egeneksitasjonen er konstant, mens lekkasjen er i prosenter av totalverdien. Implementert ved 
nVerdiForDepolarisering *= DEGRADERINGSFAKTOR\_FOR\_TIDSINTEGRASJON\_I\_NEURON.

Lekasjen er med andre ord ulineær, og sterkare des større nVerdiForDepolarisering er (som ei lekk bøtte).

\section{LTP/LTD}
LTP/LTD er \emph{synaptisk} plastisitet, men er også verdt å tenke på i forhold til neuron. Her i neuron kan det være at overordna mekanismer som bestemmer størrelsen på LTP / balansen mellom [LTP/LTD](konst. LTD, variabel LTP=>variabel syn.vekt).

Eg har lest at adtivering av dopaminreceptoren D1 legger til rette for LTP (i nokre neuron). Dette opner for å legge inn DA i Auron. 
Eg har tenkt å lage DA--kretser, som forsyner neuron med DA. No som eg har implementert egenfyring, gir dette rom for tonisk DA--fyring.
\begin{description}
	\item[tonisk fyring] Ved normal tilstand, også når forventa belønning skjer, vil DA fortsette å levere DA tonisk (rymisk, konstant)
	\item[phasic firing] Ved uventa belønning, vil DA--neurona fyre for full pupp, i periode proposjonalt med størrelsen på pos.overraskelse. Dette fører til at du auker DA--mengden i ``bøtta'', ei stund. Dette vil eg tru auker LTP i mottaker--neuronet.
	\item[pause i fyring] Ved stor skuffelse, stopper leveransen av DA i periode prop. med skuffelsen.
\end{description}

Dette har ført meg til ideen om DA--nivå i auron har direkte innverkning på grad av LTP i synapsene til neuronet. Foreløpig trur eg at det som blir påverka, er innsynapsene til neuronet.

\subsection{DA--nivå har samme effektene som for depol--verdi (sjå \ref{depolVerdi}):}
\begin{description}
	\item[Synaptisk input] Neuron får input fra DA--neuron (t.d. VTA, substantia nigra). Dette må eg ha i auron også.
	\item[Egen--eksitasjon] Neuron kan ikkje lage DA sjølv, men det som kan skje er at DA diffunderer inn i neuronet fra utsida av neuronet (endokrin overføring). Dette vurderer eg å implementere. I så fall ved å ha en peiker i kvart neuron, til extracellularFluid, og en form for overføring her. Venter med det.
	\item[Lekkasje] Også DA kan lekke, dvs. miste størrelse over tid. Dette skal implementeres i oppdaterNeuron().
\end{description}



%Skriver utgåande om synapse, (arbeidskø?), (evt. om presyn. plasticitet(syn.Vesicles) ), ++ 

\chapter{synapse}

Skrive litt om korleis det er lagt opp for constructor/destructor.
%\section{}


\section{synapse::synapse( neuron * , neuron  * , bool , float)}
a :  * \\
synapse::synapse( neuron *  pPreN\_arg, neuron  *  pPostN\_arg, bool argInhibitorisk\_effekt / * =false * /, float v / * =1 * / )\\
Constuctor. Lager synapse, med alt som trengs, deribland legger seg til i synapse-listene i pre- og post- synaptisk neuron.
\begin{description}
	\item[neuron *  pPreN\_arg] Neuron *  til presynaptisk neuron. Dette lagres i synapse::pPreNode, og som element i presyn. neurons neuron::pUtSynapser [vector]
	\item[neuron *  pPostN\_arg] Neuron *  til postsynaptisk neuron. Dette lagres i synapse::pPostNode, og som element i postsynapsens neuron::pInnsynapser
\end{description}

\section{synapse::\~synapse}
Destructor skal sørge for at synapse blir fjærnet fra presynapsens og postsynapsens ouput/input vector (de som holder styr på alle synapsene til neurona, input og output). Har slitt litt med dette i dag, men endte opp med å skrive ekstern handtering (for spesialtifellet: ~neuron -- da skal synapsen slettes (frigjøre det frie lageret), presynapsens liste over outputsynapser skal endres, postsynapsens liste skal rettes. Lagde ei løysing i ~neuron, som fungerer godt, men med litt minnelekasje (har ikkje frigjort minnet / ikkje kjørt delete() ).


			\begin{description}
				\item[ein] jejejej jelja øleskjfda øskdj ølkejrweø 
				\item[to] jlaksejør aøl dskjfø xcnv.,amnerøw
 			\end{description}




%KANN
\chapter{Synaptisk vektendring}

I neuronet er det mange synapser. Desse vil endres infinitesimalt selv for ``store'' synaptiske endringer. Dette gir at også enkeltsynapser kan sees på statistisk, når man snakker om synaptisk plastisitet.

Totalt, gir dette god grunn for å sjå på systemet fra en statistisk synsvinkel. Når det gjelder dopamin(DA), veit eg at aktivering av D1--receptoren facilitates LTP. På ein eller annan måte er m.a.o. dopamin knytta til størrelsen av synaptisk endring (f.eks. LTP). Sansynligvis er det feil, men eg velger å sjå på det som eit kar med DA, der mengden DA gir størrelsen til synaptisk vektendring (sansynligvis er det andre mekanismer som blir aktivert gjennom D1, men effekta er den samme).
Dette karret med DA lekker, omtrent som depol.--verdien. Samtidig får det meir gjennom DA--syn.--input (og kanskje endokrin diffudering).

\section{Synaptisk plastisitet--balansen}
Balansen mellom LTP(synaptisk vektauke) og LTD(synaptisk vektminke)   --- `synaptic plasticity' --- er gitt med ulike mekanismer:
\begin{description}
	\item[vektAuke] Graden av LTP er styrt av mange mekanismer, deriblant ein styrt av D1--receptoren. Kaller heile denne mekanismen for DA--mekanismen.\\
	Modellerer det med ein variabel (som eg kaller dMengdeDA\_i\_neuron), som eg vil multiplisere inn ved LTP (og dermed la LTP være proposjonal med denne variabelen.
	\item[vektMinke] Graden av LTD er sansynligvis proposjonal med synaptisk vekt(sjå s. \pageref{figur_STDP}). Kan sjå på dette som at synapsen mister postsynaptiske glutamat--receptorer, og dette er mest sansynlig at skjer ved at synapsen mister en satt mengde prosent av receptorene. M.a.o. proposjonal med syn. vekt.. 
\end{description}
Dette skaper ei likevekt mellom vektauke og vektminke -- synaptic--plasticity--equilibrium!

Denne likevekta kan forskyves bl.a. ved DA. Dette forplanter seg i pavlovian reward--mekanismer. Ved reward, vil meir DA bli levert, og synapsene vil statistisk sett auke synaptisk vekt. Ved skuffelse, vil pause i tonisk DA--leveranse oppstå, og netto ender vi med synaptisk minke i mottaker--neurona fra f.eks. VTA--neurona som leverer DA.

\subsection{Diskurs om balansen. Fram og tilbake..}
\label{diskursOmLTPLTD}
---Sjå figur \ref{figur_STDP}. \emph{Dette er kun en indikasjon, for videre undersøking!} Kansje denne figuren støtter hypotesen min. Tenk meir på denne( merk: \emph{``relative change''}, og den kvantitative følelsen i figuren ).
\begin{quote}
In addition, the amount of potentiation decreases for stronger synapses, whereas the relative amount of depression is independent of synaptic size.
\end{quote}
Eg trur ikkje det er heilt klart for denne forfatteren. Han blander absolutt mål(først), med relativt mål på slutten.

Det er ivertfall noke her. Fig-teksten i \ref{figur_STDP} seier at det er relative LTP som minker ved mykje synaptisk vekt. LTD har konstant relativ syn.vektstap. Tenk meir på dette!
\begin{itemize}
	%\item Kanskje LTP varierer som tidligere antall, minus en syn.vekt--varierende faktor?
	\item Ei tolkning er at LTP ikkje er avhengig av syn.vekt. \emph{`relative change'}. Minker selvsagt, dersom endringa er konstant, selv om syn.vekt er stor. Dette gir i så fall `relative change' som er liten.\\
--- I denne tolkninga er det at `relative change' ved LTD ikkje blir mindre når syn.vekt blir stor, som er bemerkelsesverdig!
\end{itemize}
%figur om STDP. Viktig i forhold til LTP/LTD balansen.
\begin{figure}[!htbp]
	\label{figur_STDP}
	\centering
	\includegraphics[width=0.8\textwidth]{figurSTDP.jpeg}
	\caption{Spike timing-dependent plasticity. a, Synapses are potentiated if the synaptic event precedes the postsynaptic spike. Synapses are depressed if the synaptic event follows the postsynaptic spike. b, The time window for synaptic modification. The relative amount of synaptic change is plotted versus the time difference between synaptic event and the postsynaptic spike. The amount of change falls off exponentially as the time difference increases. In addition, the amount of potentiation decreases for stronger synapses, whereas the relative amount of depression is independent of synaptic size.}
\end{figure}


\section{Tap av ``DA''}
Det er ikkje for stor tvil om at ``DA'' er viktig element for størrelsen av syn.endring. Vi får det ved syn.overføring og kanskje ved diffundering (sjå \ref{DA_mekanismer}), men korleis tapes det? Korleis mister vi ``DA'' fra neuronet?
\begin{itemize}
	\item Ved ``lekkasje'' ?
	\item Blir ``brukt opp'' ved LTP (og dermed kontinuerlig; siden vi har kontinuerlig LTD, må vi også ha kontinuerlig LTP for å opprettholde syn. vekt. Dette gir for såvidt eit resultat som forrige punkt).
\end{itemize}
Andre alternativ resulterer for så vidt i også i effekt som den første(lekkasje), så kan implementere begge to, for å minke prosessorkraft (bedre å bare miste DA, enn å bruke det til LTP -- så ta vekk dette vekt--tillegget samme runde..


...\\
Meir\\
...


\section{Operant--betinging}
Klassisk betinging(pavlov) er ganske grei å sjå for seg mekanismane bak(på veldig lavt nivå, minnelause system, f.eks. snegler), men kva med operantbetinging? Dette er betinging, der systemet ser fram til (bevisst eller ubevisst) eit framtidig mål\footnote{Kan også være det framtidige målet å gjøre noken andre glad; viktig funksjon for flokkdyr..}.
Når dette målet er nådd, vil belønning (i form av DA til de aktuelle kretsane) gjennomføres.

Operant--betinging er som sagt mykje meir kompleks form for betinging, som kun høgare utvikla (patte?--) dyr har. Krever en viss form for executive functions (planlegging, sekvesiering, målretta handlinger, ...). Videre trengs eit lenger tids--minne, for å belønne huske mål/handlingane når de er gjennomført. 

\textbf{\emph{Tenk på dette}}, når meir utarbeida modeller for minne/ belønning er på vei!






%\chapter{Tid}
%Når det gjelder tid, så har eg også tatt med dette i universet mitt (simuleringsuniverset). Neurona/synapsene er tidsvariante (bruke tid for å lades opp). Denne tida er ikkje avhengig av processor-tics, eller UTC, men av andre ting i simuVerset eg har laga. Løysinga mi var å lage en scheduler, som også lager discrete tidsiterering.
%\section{Scheduleren}
%I utgangspunktet var planen å lage eit system for sekvensen / kausaliteten i Auron-nettet mitt. Møtte raskt på problematikk med å la alle neurona som skulle fyre før neuron A, fyre, før A.

%Foreløpig løysing er basert på ideen om to køer, der kjøringa av de to køene altererer, når ei kø er ferdigkjørt. Køene inneholder jobber, i form av synapse*. Alle arvtakera etter class synapse har funksjonen aktiviserOgRegnUt(). 
%\begin{itemize}
%	\item I vanlige synapser, vil dette føre til:
%		\\ - oppdatering av presyn. og postsyn. neuron ( oppdatering av [depolarisasjon, DA, anna] på grunnlag av tid siden sist oppdatering), 
%		\\ - synaptisk overføring -> EPSP / IPSP
%		\\ - LTP/LTD (synaptic plasticity)
%	\item I synSkilleElement er aktiviserOgRegnUt overlagra, og kjører isteden en funk som: 
%		\\ - itererer tid. 
%		\\ - oppdaterer alle synapser i pNesteSynapseSomIkkjeErFerdigOppdatert.
%		\\ - holder orden på alle neuron i pNesteFyringForNeuron. Dersom ett er ført opp til å fyre neste iterasjon, legges det til i pNesteSynapseUtregningsKoe (og vil fyre på slutten av neste iterasjon).
%\end{itemize}



\chapter{Variabler og funksjoner - global}
\section{global:}
	\begin{description}
		\item[void* arbeidsKoeArbeider(void*)] - \textit{funksjon} \\
			Har ansvaret for å holde arbeidskø i gang (og iterere tid når synSkilleElement kalles). 
		\item[ulTidsiterasjoner] - \textit{extern unsigned long} \\
			Tidspunkt iterert ved synSkilleElement . Sjå "TID" øverst i denne fila.
	\end{description}

\pagebreak




\chapter{Variabler og funksjoner - neuron}
	Class neuron - introdusksjon. Her..

	\section{Variabler:}
		\subsection{protected}
			\begin{description}
				\item[nVerdiForDepolarisering] : \textit{int} 		  \hspace{3cm} protected \\
				Verdi for depolarisering av neuron. Brukes til å kalkulere om neuron fyrer. 
				\item[ulTimestampForrigeInput] : \textit{unsigned long} \hspace{5cm}	protected \\
				Tidspunkt for forrige input ved den globale tidsvariablen, som blir iterert i synSkilleElement. (sjå "TID" øverst i fila).
				Brukes til å regne ut degradering av nVerdiForDepolarisering etter tid, når nVerdiForDepolarisering skal brukes til å beregne noke.
				\item[ulTimestampFyring]  	 : \textit{unsigned long} \hspace{5cm}	protected \\
				Tidspunkt for forrige fyring av aksjonspotensial. (sjå "TID" øverst i fila)\\
				Planlagt bruk er ved STDP og LTP/D.
				
				\item[navn] 			 : \textit{std::string} \hspace{4cm}	protected\\
				navn, i bruk ved utskrift.
			
				\item[uTerskel] 		  	:  \textit{unsigned}  	\hspace{3cm}	protected\\
				Terskel for fyring av aksjonspotensial. Settes i konstruktor, og default satt til TERSKEL\_DEFAULT (akkurat no er denne satt til 100)
				
				\item[uTotaltAntallReceptoreIPostsynNeuron\_setpunkt] : \textit{unsigned} \indent - protected \\
				Planlagt setpkt for antall receptore. Ikkje i gang enda. Akkurat no har eg overgang fra int antallRecepore til float.\\
				// ta med egenfrekvens også, etterkvart. for now skal denne være konst.
 			\end{description}
		\subsection{public}
			\begin{description}
				\item[pUtSynapser og pInnSynapser] : \textit{std::vector <synapse*>} \indent - public \\
				Holder styr på utsynapser (for å legge til i arbeidskø, når neuron fyrer), og innsynapser (for å utføre hetero-LTD og LTP) 
				Kanskje \'{o}g STDP.
	\end{description}

	\section{Funksjoner:  -  	public}
		
		\begin{description}
			\item[neuron\=( std::string n )]   : \textit{constructor} 	- public \\
				arg : 	std::string \\
				retur : 	- \\
				
				Constructor for neuron.
				
			\item[void settInnElektrisitetGjennomProbe( )] 	 		- public 	\\
				arg: 	void 							\\
				retur: 	void 						\\
				funksjon: 	Fører til at neuron fyrer. (For å indusere eit signal, for tesing av neuron..) \\
		
		
			\item[virtual void kalkulerDegraderingAvVerdiPaaGrunnlagAvTimestampSidenSist( )] 			- protected - inIine
			\item[virtual int sendInnPostsynaptiskEksitatoriskEllerInhibitoriskSignal( int nInnsignalArg )] - protected - inIine
			\item[virtual int fyr()] 												- protected - inIine
			\item[virtual int sjekkKorDepolarisertNeuronEr( )] 								- protected - inIine
 		\end{description}
	\section{FRIEND:}
		\begin{itemize}
			\item	class synapse
			\item operator <<( , neuron)
		\end{itemize}


\pagebreak






\chapter{class synapse:}

	Eg trur desse beskrivelsane er litt utdatert. Lenge siden eg har skrevet på det. Holder på med neuron, no.


	\section{Variabler:}
		\begin{description}
			\item[bInhibitorisk\_effekt] 	 		: 	\textit{const bool} 	- 	private \\
		konstant bool som blir initiert ved konstruering av synapse. Bestemmer om synapse er eksitatorisk eller inhibitorisk. Må initieres i constr.
		Dersom denne er true, har synapsen inhibitorisk effekt.
		\item[ulTimestampForrigeOppdatering] 		: 	\textit{unsigned long}  - 	private \\
		Tids-signal for oppdatering av synaptic vesicles (kanskje også for LTP/D, etterkvart)
		
		\item[ulAntallSynapticVesiclesAtt] 			: 	\textit{unsigned long} 	- 	private \\
	        unsignedlong som holder styr på kor mange synaptic vesicles som er igjen i synapsen. Siden eg modellerer det som ei vansøyle/opning nedst,
		er slepp av syn.vesicles avhengig av kor mange som er igjen. (prosentvis slepp) Dette gir også gradvis depression.
 			\begin{itemize}
				\item Opplading av desse skjer ved to mekanismer i oppdater()
				\item snBestilltSynteseAvSVFraForrigeIter : Bestillt antall syntese av SV, fra forrige iter. Denne variabelen er smooth-a litt
				    i tid, vha. ei MA-filter-lignande effekt. ( verdi+=ny, verdi/=2; )
			 	\item snBestilltReproduksjonAvSvFraForrigeIter : Repoduksjon av S.V. som ligger inne i membran. Denne er ikkje smooth-a, da det 
			       	    kunne føre til stygge effekter av unsigned sv-i-membran.
			\end{itemize}    

		\item[fGlutamatReceptoreIPostsynMem] 		: 	\textit{float} 		- 	private \\
		Effekt (postsyn. invirkning) ved kvar synaptic vesicle. Viktig element i langtids plastisitet i synapsene. Spesielt LTP og heteroLTD.

	%/*	- dOppladingsFartForSynVesicles : 	double  	: 	private
	%	Mi hypotese: dersom mi hypotese stemmer, så skal det være litt treighet i systemet for opplading av synaptic vesicles. Dette gjør eg ved
	%	å bruke forrige iterasjons oppladingsfart (eller FIRvariant). Dette gjør at vi får oversving av synaptic vesicles. Dette fører også til
	%        potentiation.
	%	/ *Dette er ikkje implementert enda */	

		\item[ulAntallSynV\_setpunkt] 			: 	\textit{unsigned long} 	- 	private \\
		Mi hypotese: I stadenfor at effekta over står åleine, kan vi innføre variabelt setpunkt for antall synaptic vesicles. Denne kan økes når:
		presyn.terminal over tid har lite synaptic vesicles. || når presyn.term. har lite s.v. , og postsyn. er sterkt depol || anna?
		Dette kan bidra til lengre tids augentation. Og kanskje litt på LTP (på hetero-måten).
		
		\item[pPreNode] 						: 	\textit{neuron*}  	-  	public \\
			Peiker på presynaptisk neuron (før synapsen).
		\item[pPostNode] 	 	 				: 	\textit{neuron*} 		-   	public \\
			Peiker på postsynaptisk neuron (etter synapsen).
	\end{description}


	\section{Funksjoner:}
	\begin{description} 
	 	\item[synapse(char c)] : bInhibitorisk\_effekt(false)  	: 	constructor 	:  	public \\

	 	\item[synapse( neuron* pPreN, neuron* pPostN, bool argInhib =0, float v =1)] 	: \textit{constructor} 	public \\
 		arg:        fra-neuron,     til-neuron, inhibitorisk??? , glutamatreceptore \\
		Gjør standard konstruktorgreier, men i tillegg legger den synapse* til seg sjølv til, i presyn. si pUtSynapser-liste, og i postsyn. si 
		pInnSynapser-liste. På denne måten holder den automagisk rede på neuron. 
	
		\item[void oppdater()] 		: 	void (void) 	: 	private\\
		Funksjon for opplading av synaptic vesicles. Dette er skal skje kvar tidsiterasjon. Har egen kø for dette: 
		synapse::pNesteSynapseSomIkkjeErFerdigOppdatert\_Koe. Alle element før synSkilleElement, i denne, skal sjekkes vha. funksjonen oppdater() 
		Dette gjøres i synSkilleElement::aktiviserOgRegnUt(), kalt fra tidskøa ( pNesteSynapseUtregningsKoe ).
		Vanlige synapser's oppdater() returnerer 1, mens synSkilleElement::oppdater() returnerer 0. Dette for å kunne gjøre slik: 
		while(kø->oppdater() ); Da vil skilleElementet kunne skille oppdateringskøa.

		\item[virtual void aktiviserOgRegnUt()] : 	\textit{virtual void(void)} : 	public \\
		Funksjon som kalles fra neuron, når den fyrer. Regner ut overføringa av signal til postsyn. neuron. Har tatt med synaptic vesicle, med 
		antall, syntese, regenerering fra membran, for å få med potentation-effekt (teori nr.1 ). Har ikkje tatt med teori nr 2, om at slepping
		skjer som funk av areal.
		Grunnen til at den er virtual, er at den overlagres i synSkilleElement, der den gjør alt som skal gjøres mellom kvar tids-iterasjon.
		Det som er med på å bestemme overføring av signal er:
			- antall synaptic vesicles i presyn. (har eit heilt maskineri for å ligne bioNeuron)
			- antall glutamatreceptorer i postsyn mem. (ikkje ferdig implementert. Planlagt viktig i LTP og LTD)
		- Det er her hovedfunksjonen til synapse regnes ut, som kalkulering av antall syn.v. som skal sleppes, kor mange som er igjen, om effekta er
		eksitatorisk eller inhibitorisk, oppdatering av antall s.v., osv.
	\end{description}




	\section{FRIEND}
		\begin{description}
			\item[operator<<( , synapse )]
			\item[synSkilleElement : public synapse]
		\end{description}
	


Fotnoter\footnote{dette er fotnote} lages slik: \\footnote{Dette er fotnote}


%\bibliography{bibliografi}
%\bibliographystyle{plain}
\end{document}

