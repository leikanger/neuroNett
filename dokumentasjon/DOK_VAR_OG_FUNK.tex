
%KANN

\chapter{Variabler og funksjoner - global}
\section{global:}
	\begin{description}
		\item[void* arbeidsKoeArbeider(void*)] - \textit{funksjon} \\
			Har ansvaret for å holde arbeidskø i gang (og iterere tid når synSkilleElement kalles). 
		\item[ulTidsiterasjoner] - \textit{extern unsigned long} \\
			Tidspunkt iterert ved synSkilleElement . Sjå "TID" øverst i denne fila.
	\end{description}

\pagebreak




\chapter{Variabler og funksjoner - neuron}
	Class neuron - introdusksjon. Her..

	\section{Variabler:}
		\subsection{protected}
			\begin{description}
				\item[nVerdiForDepolarisering] : \textit{int} 		  \hspace{3cm} protected \\
				Verdi for depolarisering av neuron. Brukes til å kalkulere om neuron fyrer. 
				\item[ulTimestampForrigeInput] : \textit{unsigned long} \hspace{5cm}	protected \\
				Tidspunkt for forrige input ved den globale tidsvariablen, som blir iterert i synSkilleElement. (sjå "TID" øverst i fila).
				Brukes til å regne ut degradering av nVerdiForDepolarisering etter tid, når nVerdiForDepolarisering skal brukes til å beregne noke.
				\item[ulTimestampFyring]  	 : \textit{unsigned long} \hspace{5cm}	protected \\
				Tidspunkt for forrige fyring av aksjonspotensial. (sjå "TID" øverst i fila)\\
				Planlagt bruk er ved STDP og LTP/D.
				
				\item[navn] 			 : \textit{std::string} \hspace{4cm}	protected\\
				navn, i bruk ved utskrift.
			
				\item[uTerskel] 		  	:  \textit{unsigned}  	\hspace{3cm}	protected\\
				Terskel for fyring av aksjonspotensial. Settes i konstruktor, og default satt til TERSKEL\_DEFAULT (akkurat no er denne satt til 100)
				
				\item[uTotaltAntallReceptoreIPostsynNeuron\_setpunkt] : \textit{unsigned} \indent - protected \\
				Planlagt setpkt for antall receptore. Ikkje i gang enda. Akkurat no har eg overgang fra int antallRecepore til float.\\
				// ta med egenfrekvens også, etterkvart. for now skal denne være konst.
 			\end{description}
		\subsection{public}
			\begin{description}
				\item[pUtSynapser og pInnSynapser] : \textit{std::vector <synapse*>} \indent - public \\
				Holder styr på utsynapser (for å legge til i arbeidskø, når neuron fyrer), og innsynapser (for å utføre hetero-LTD og LTP) 
				Kanskje \'{o}g STDP.
	\end{description}

	\section{Funksjoner:  -  	public}
		
		\begin{description}
			\item[neuron\=( std::string n )]   : \textit{constructor} 	- public \\
				arg : 	std::string \\
				retur : 	- \\
				
				Constructor for neuron.
				
			\item[void settInnElektrisitetGjennomProbe( )] 	 		- public 	\\
				arg: 	void 							\\
				retur: 	void 						\\
				funksjon: 	Fører til at neuron fyrer. (For å indusere eit signal, for tesing av neuron..) \\
		
		
			\item[virtual void kalkulerDegraderingAvVerdiPaaGrunnlagAvTimestampSidenSist( )] 			- protected - inIine
			\item[virtual int sendInnPostsynaptiskEksitatoriskEllerInhibitoriskSignal( int nInnsignalArg )] - protected - inIine
			\item[virtual int fyr()] 												- protected - inIine
			\item[virtual int sjekkKorDepolarisertNeuronEr( )] 								- protected - inIine
 		\end{description}
	\section{FRIEND:}
		\begin{itemize}
			\item	class synapse
			\item operator <<( , neuron)
		\end{itemize}


\pagebreak






\chapter{variabler og funksjoner -- synapse:}

	Eg trur desse beskrivelsane er litt utdatert. Lenge siden eg har skrevet på det. Holder på med neuron, no.


	\section{Variabler:}
		\begin{description}
			\item[bInhibitorisk\_effekt] 	 		: 	\textit{const bool} 	- 	private \\
		konstant bool som blir initiert ved konstruering av synapse. Bestemmer om synapse er eksitatorisk eller inhibitorisk. Må initieres i constr.
		Dersom denne er true, har synapsen inhibitorisk effekt.
		\item[ulTimestampForrigeOppdatering] 		: 	\textit{unsigned long}  - 	private \\
		Tids-signal for oppdatering av synaptic vesicles (kanskje også for LTP/D, etterkvart)
		
		\item[ulAntallSynapticVesiclesAtt] 			: 	\textit{unsigned long} 	- 	private \\
	        unsignedlong som holder styr på kor mange synaptic vesicles som er igjen i synapsen. Siden eg modellerer det som ei vansøyle/opning nedst,
		er slepp av syn.vesicles avhengig av kor mange som er igjen. (prosentvis slepp) Dette gir også gradvis depression.
 			\begin{itemize}
				\item Opplading av desse skjer ved to mekanismer i oppdater()
				\item snBestilltSynteseAvSVFraForrigeIter : Bestillt antall syntese av SV, fra forrige iter. Denne variabelen er smooth-a litt
				    i tid, vha. ei MA-filter-lignande effekt. ( verdi+=ny, verdi/=2; )
			 	\item snBestilltReproduksjonAvSvFraForrigeIter : Repoduksjon av S.V. som ligger inne i membran. Denne er ikkje smooth-a, da det 
			       	    kunne føre til stygge effekter av unsigned sv-i-membran.
			\end{itemize}    

		\item[fGlutamatReceptoreIPostsynMem] 		: 	\textit{float} 		- 	private \\
		Effekt (postsyn. invirkning) ved kvar synaptic vesicle. Viktig element i langtids plastisitet i synapsene. Spesielt LTP og heteroLTD.

	%/*	- dOppladingsFartForSynVesicles : 	double  	: 	private
	%	Mi hypotese: dersom mi hypotese stemmer, så skal det være litt treighet i systemet for opplading av synaptic vesicles. Dette gjør eg ved
	%	å bruke forrige iterasjons oppladingsfart (eller FIRvariant). Dette gjør at vi får oversving av synaptic vesicles. Dette fører også til
	%        potentiation.
	%	/ *Dette er ikkje implementert enda */	

		\item[ulAntallSynV\_setpunkt] 			: 	\textit{unsigned long} 	- 	private \\
		Mi hypotese: I stadenfor at effekta over står åleine, kan vi innføre variabelt setpunkt for antall synaptic vesicles. Denne kan økes når:
		presyn.terminal over tid har lite synaptic vesicles. || når presyn.term. har lite s.v. , og postsyn. er sterkt depol || anna?
		Dette kan bidra til lengre tids augentation. Og kanskje litt på LTP (på hetero-måten).
		
		\item[pPreNode] 						: 	\textit{neuron*}  	-  	public \\
			Peiker på presynaptisk neuron (før synapsen).
		\item[pPostNode] 	 	 				: 	\textit{neuron*} 		-   	public \\
			Peiker på postsynaptisk neuron (etter synapsen).
	\end{description}


	\section{Funksjoner:}
	\begin{description} 
	 	\item[synapse(char c)] : bInhibitorisk\_effekt(false)  	: 	constructor 	:  	public \\

	 	\item[synapse( neuron* pPreN, neuron* pPostN, bool argInhib =0, float v =1)] 	: \textit{constructor} 	public \\
 		arg:        fra-neuron,     til-neuron, inhibitorisk??? , glutamatreceptore \\
		Gjør standard konstruktorgreier, men i tillegg legger den synapse* til seg sjølv til, i presyn. si pUtSynapser-liste, og i postsyn. si 
		pInnSynapser-liste. På denne måten holder den automagisk rede på neuron. 
	
		\item[void oppdater()] 		: 	void (void) 	: 	private\\
		Funksjon for opplading av synaptic vesicles. Dette er skal skje kvar tidsiterasjon. Har egen kø for dette: 
		synapse::pNesteSynapseSomIkkjeErFerdigOppdatert\_Koe. Alle element før synSkilleElement, i denne, skal sjekkes vha. funksjonen oppdater() 
		Dette gjøres i synSkilleElement::aktiviserOgRegnUt(), kalt fra tidskøa ( pNesteSynapseUtregningsKoe ).
		Vanlige synapser's oppdater() returnerer 1, mens synSkilleElement::oppdater() returnerer 0. Dette for å kunne gjøre slik: 
		while(kø->oppdater() ); Da vil skilleElementet kunne skille oppdateringskøa.

		\item[virtual void aktiviserOgRegnUt()] : 	\textit{virtual void(void)} : 	public \\
		Funksjon som kalles fra neuron, når den fyrer. Regner ut overføringa av signal til postsyn. neuron. Har tatt med synaptic vesicle, med 
		antall, syntese, regenerering fra membran, for å få med potentation-effekt (teori nr.1 ). Har ikkje tatt med teori nr 2, om at slepping
		skjer som funk av areal.
		Grunnen til at den er virtual, er at den overlagres i synSkilleElement, der den gjør alt som skal gjøres mellom kvar tids-iterasjon.
		Det som er med på å bestemme overføring av signal er:
			- antall synaptic vesicles i presyn. (har eit heilt maskineri for å ligne bioNeuron)
			- antall glutamatreceptorer i postsyn mem. (ikkje ferdig implementert. Planlagt viktig i LTP og LTD)
		- Det er her hovedfunksjonen til synapse regnes ut, som kalkulering av antall syn.v. som skal sleppes, kor mange som er igjen, om effekta er
		eksitatorisk eller inhibitorisk, oppdatering av antall s.v., osv.
	\end{description}




	\section{FRIEND}
		\begin{description}
			\item[operator<<( , synapse )]
			\item[synSkilleElement : public synapse]
		\end{description}
	
